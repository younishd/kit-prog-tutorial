%% LaTeX-Beamer template for KIT design
%% by Erik Burger, Christian Hammer
%% title picture by Klaus Krogmann
%%
%% version 2.1
%%
%% mostly compatible to KIT corporate design v2.0
%% http://intranet.kit.edu/gestaltungsrichtlinien.php
%%
%% Problems, bugs and comments to
%% burger@kit.edu

\documentclass[18pt]{beamer}

%% SLIDE FORMAT

% use 'beamerthemekit' for standard 4:3 ratio
% for widescreen slides (16:9), use 'beamerthemekitwide'

\usepackage{templates/beamerthemekit}
% \usepackage{templates/beamerthemekitwide}

\usepackage[utf8]{inputenc}
\usepackage{hyperref}
\usepackage{listings}
\usepackage{color}
%\usepackage{xcolor}
%\usepackage{colortbl}
%\usepackage{array}
%\usepackage{tikz}
%\usetikzlibrary{calc,shapes.multipart,chains,arrows}

%\definecolor{lime}{HTML}{8FFF53}

\newcommand{\quotes}[1]{``#1''}

%% TITLE PICTURE

% if a custom picture is to be used on the title page, copy it into the 'logos'
% directory, in the line below, replace 'mypicture' with the
% filename (without extension) and uncomment the following line
% (picture proportions: 63 : 20 for standard, 169 : 40 for wide
% *.eps format if you use latex+dvips+ps2pdf,
% *.jpg/*.png/*.pdf if you use pdflatex)

\titleimage{greendrop}

%% TITLE LOGO

% for a custom logo on the front page, copy your file into the 'logos'
% directory, insert the filename in the line below and uncomment it

%\titlelogo{mylogo}

% (*.eps format if you use latex+dvips+ps2pdf,
% *.jpg/*.png/*.pdf if you use pdflatex)

%% TikZ INTEGRATION

% use these packages for PCM symbols and UML classes
% \usepackage{templates/tikzkit}
% \usepackage{templates/tikzuml}

% the presentation starts here

\title[Exceptions]{Programmieren:\\ Exceptions}
\subtitle{Tutorium 30}
\author{YouniS Bensalah}
\date{December 18, 2015}

\institute{Chair for Software Design and Quality}

% Bibliography

\usepackage[citestyle=authoryear,bibstyle=numeric,hyperref,backend=biber]{biblatex}
\addbibresource{templates/example.bib}
\bibhang1em

\begin{document}

% change the following line to "ngerman" for German style date and logos
\selectlanguage{english}

%title page
\begin{frame}
\titlepage
\end{frame}

%table of contents
\begin{frame}{Heute}
\tableofcontents
\end{frame}

\section{Organisatorisches}

\begin{frame}{Termine}
    \textbf{Vorlesung}
    \begin{itemize}
        \item Die Vorlesung am 23.12.2015 entfällt.
        \item Die erste Vorlesung 2016 findet am 13.01.2016 statt.
    \end{itemize}

    \textbf{Tutorien}
    \begin{itemize}
        \item Die Tutorien finden im Jahr 2015 bis zum 22.12.2015 statt.
        \item Die Tutorien im Jahr 2016 beginnen ab dem 13.01.2016.
    \end{itemize}
\end{frame}

\section{Exceptions}

\begin{frame}{Exceptions}

\end{frame}

\begin{frame}{Fragen ?}
    \begin{figure}
        \includegraphics[scale=.5]{img/question_to_idea.jpg}
    \end{figure}
\end{frame}

\section{Programmieraufgabe}

\begin{frame}{Programmieraufgabe}
    \Large{Enough theory (;}
\end{frame}

\begin{frame}{Collatz}
    Das war ja zu einfach\dots
\end{frame}

\begin{frame}{Super List 9K}
    \textbf{Super List 9K} soll eine verbesserte Version der bereits bekannten \textit{verketteten Liste} werden.

    \begin{itemize}
        \item \textbf{doppelt verkettete Liste}
        \item \texttt{addFirst()} und \texttt{addLast()} fügen ein Element an den Anfang bzw. das Ende der Liste ein.
        \item \texttt{remove()} löscht alle Elemente aus der Liste, die mit der \texttt{equals()}-Methode gleich einem gegebenen Element sind.
        \item \texttt{contains()} prüft nach, ob ein Element in der Liste vorhanden ist.
        \item \texttt{size()} gibt die Länge der Liste (Anzahl Elemente) aus.
        \item \texttt{count()} zählt, wie oft ein Element in der Liste vorkommt.
        \item \textbf{Abstrakter Datentyp}
        \item \textbf{Generics !}
    \end{itemize}

\end{frame}

\begin{frame}[fragile]{Super List 9K}
    \begin{enumerate}
        \item Download and unzip the template\\ \url{http://younishd.fr/prog/superlist9k-template.zip}
        \item Think\dots
        \item Code\dots
        \item Compile and test:\\
        \begin{lstlisting}[language=Java,basicstyle=\scriptsize]
% javac superlist9k/SuperCell.java \\
        superlist9k/SuperList.java \\
        superlist9k/SuperListInterface.java \\
        test/SuperTest.java

% java test.SuperTest
        \end{lstlisting}
        \item Repeat \texttt{2-4} until all tests are \textcolor{green}{green}!
    \end{enumerate}
\end{frame}

\begin{frame}{Super List 9K}
    Die Lösung findet Ihr hier:
    \begin{itemize}
        \item \url{http://younishd.fr/prog/superlist9k.zip}
    \end{itemize}
\end{frame}

\appendix
\beginbackup

\begin{frame}{Bis nächstes Jahr !}
    \begin{figure}
        \includegraphics[scale=.25]{img/java.jpg}
    \end{figure}
\end{frame}

\backupend

\end{document}
