%%
%% kit-prog-tutorial
%%
%% Slides for my Java programming tutorial at KIT using LaTeX beamer.
%%
%% Copyright (c) 2015-2016 YouniS Bensalah <younis.bensalah@gmail.com>
%%
%% This work is released to the public domain.
%% For the full copyright and license information, please view the LICENSE file.
%%

\documentclass[18pt]{beamer}

\usepackage{templates/beamerthemekit}

\usepackage[utf8]{inputenc}
\usepackage{hyperref}
\usepackage{listings}

\titleimage{road}

\newcommand{\tagline}{A closer look at Methods and Constructors}
\newcommand{\quotes}[1]{``#1''}

\title[Programmieren\hspace{2.5pt}--\hspace{2.5pt}\tagline]{\tagline}
\subtitle{Programmieren~\textbar~Tutorium 32}

\author{YouniS Bensalah}
\date{21. November 2016}

\institute{Chair for Software Design and Quality}

\usepackage[citestyle=authoryear,bibstyle=numeric,hyperref,backend=biber]{biblatex}
\addbibresource{templates/example.bib}
\bibhang1em

\begin{document}

% remove annoying figure prefix in caption
\setbeamertemplate{caption}{\raggedright\insertcaption\par}

\selectlanguage{english}

\begin{frame}
    \titlepage
\end{frame}

% \begin{frame}{Heute}
%     \tableofcontents
% \end{frame}

\section{Organisatorisches}

\begin{frame}{Prüfungsanmeldung}
    \begin{itemize}
        \item Im \textbf{Studiportal} anmelden
        \begin{itemize}
            \item \texttt{https://campus.studium.kit.edu}

        \end{itemize}
    \end{itemize}
    \begin{itemize}
        \item \textbf{CAS} $\rightarrow$ Anmeldung online (Info, InWi)
        \item \textbf{HISPOS} $\rightarrow$ Anmeldung entweder online oder über blauen Schein
        \begin{itemize}
            \item \alert{Sehr zeitnah im Studienbüro vorbeigehen!}
        \end{itemize}
    \end{itemize}
\end{frame}

\begin{frame}{Praktomat}
    \begin{itemize}
        \item Im \textbf{Praktomat} anmelden
        \begin{itemize}
            \item \url{https://praktomat.cs.kit.edu/2016_WS}
        \end{itemize}
        \item \alert{\textbf{Da gebt ihr die Übungsaufgaben ab}}
        \item Login via KIT-Account (uxxxx)
        \item Zugang von außerhalb über VPN
        \begin{itemize}
            \item \url{https://www.scc.kit.edu/dienste/vpn.php}
        \end{itemize}
    \end{itemize}

\end{frame}

\begin{frame}{ILIAS}
    \begin{itemize}
        \item Im \textbf{ILIAS} anmelden
        \begin{itemize}
            \item \url{https://ilias.studium.kit.edu}
        \end{itemize}
        \item \textbf{Forum zu Vorlesung und Übungsaufgaben}
        \item Login via KIT-Account (uxxxx)
        \item \alert{Kurs \textit{Programmieren} beitreten!}
    \end{itemize}
\end{frame}



\section{Methoden}

\begin{frame}{Methoden}
    \begin{itemize}
        \item \textbf{Methoden beschreiben das Verhalten von Objekten}
        \item Alle Objekte vom gleichen Typ können die gleichen Funktionen ausführen
        \begin{itemize}
            \item \textit{Alle Ampeln können von Rot auf Grün schalten}
        \end{itemize}
        \vspace{.2in}
        \item Methoden können\dots
        \begin{itemize}
            \item Berechnungen durchführen
            \item Werte züruckgeben
            \item den Zustand eines Objekts verändern
        \end{itemize}
    \end{itemize}
\end{frame}

\begin{frame}{This}
    \begin{itemize}
        \item \Huge{\texttt{this}}
        \begin{itemize}
            \item Referenz auf aktuelles Objekt
            \item Vordefinierte Variable
            \item Sichtbar in jeder (nicht-statischen) Methode
            \item Erlaubt Zugriff auf Attribute und Methoden des aktuellen Objekts
            \item Explizite Unterscheidung zwischen Attributen und lokalen Variablen
        \end{itemize}
    \end{itemize}
\end{frame}

\begin{frame}[fragile]{This}
    \begin{exampleblock}{}
        \begin{lstlisting}[language=Java,basicstyle=\Huge]
this.name
        \end{lstlisting}
    \end{exampleblock}
    \begin{itemize}
        \item Das Attribut \texttt{name} von \textbf{DIESER} Instanz
    \end{itemize}
\end{frame}


\begin{frame}[fragile]{This}
    \begin{exampleblock}{}
        \begin{lstlisting}[language=Java]
public void sayName() {
    System.out.println(this.name);
}
        \end{lstlisting}
    \end{exampleblock}
\end{frame}

\begin{frame}{Tipp 17: Always write \texttt{this}}
    \begin{itemize}
        \item Immer explizit \texttt{this} schreiben!
        \item Code ist sofort klarer
        \item Keine Verwechslungsgefahr zwischen Attributen und lokalen Variablen
        \item Vermeidet unabsichtliche Überdeckung von Attributen durch lokale Variablen
    \end{itemize}
\end{frame}

\begin{frame}{Tipp 17: Always write \texttt{this}}
    \center
    \Huge{Just do it!}
\end{frame}

\begin{frame}{Parameter}
    \begin{itemize}
        \item Ein \textbf{Parameter}\dots
        \begin{itemize}
            \item ist eine Variable
            \item enthält an Methode übergebenen Wert
        \end{itemize}
        \vspace{.2in}
        \item Eine Methode kann 0, 1 oder mehrere Parameter haben
    \end{itemize}
\end{frame}

\begin{frame}{Formaler vs. aktueller Parameter}
    \begin{itemize}
        \item \textbf{Formaler Parameter}
        \begin{itemize}
            \item Bezeichner, der in einer Methode verwendet wird
            \item $x$ ist formaler Parameter in $f(x) := e^x$
        \end{itemize}
        \vspace{.2in}
        \item \textbf{Aktueller Parameter}
        \begin{itemize}
            \item Tatsächlicher Wert, der an die Methode durch den Aufrufer übergeben wird
            \item $42$ ist aktueller Parameter in $f(42)$
        \end{itemize}
    \end{itemize}
\end{frame}

\begin{frame}[fragile]{Formaler vs. aktueller Parameter}
    \begin{exampleblock}{}
        \begin{lstlisting}[language=Java]
// something: formaler parameter
public void say(String something) {
    System.out.println(something);
}
        \end{lstlisting}
    \end{exampleblock}
    \begin{exampleblock}{}
        \begin{lstlisting}[language=Java]
// name: aktueller parameter
String name = "Steve";
student.say(name);
        \end{lstlisting}
    \end{exampleblock}
\end{frame}

\begin{frame}[fragile]{Rückgabewert}
    \begin{itemize}
        \item Eine Methode \textit{kann} einen Rückgabewert liefern
        \item Rückgabetyp \texttt{void} $\rightarrow$ \textbf{kein} Rückgabewert
    \end{itemize}
    \begin{exampleblock}{}
        \begin{lstlisting}[language=Java]
public int getHeight() {
    return this.height;
}

public void setWidth(int w) {
    this.width = w;
}
        \end{lstlisting}
    \end{exampleblock}
\end{frame}

\section{Konstruktoren}

\begin{frame}{Konstruktoren}
    \begin{block}{}
        \begin{itemize}
            \item Ein \textbf{Konstruktor} ist eine spezielle \textbf{Methode}, die beim \textbf{Erstellen eines neuen Objekts} (\texttt{new}) aufgerufen wird
            \item \textbf{Attribute} sollen \textbf{initialisiert} werden
            \item Das neue Objekt startet in \textbf{gültigem Zustand}
        \end{itemize}
    \end{block}


\end{frame}

\begin{frame}{Default-Konstruktor}
    \begin{block}{}
        \begin{itemize}
            \item Java stellt für jede Klasse zunächst einen \textbf{Default-Konstruktor} zur Verfügung
            \item Keine Argumente
            \item Alle Attribute werden mit \textit{Null} initialisiert
        \end{itemize}
    \end{block}



\end{frame}

\begin{frame}[fragile]{Default Values}
    \textit{Null?}
    \vspace{.2in}
    \begin{itemize}
        \item \texttt{byte}, \texttt{short}, \texttt{int} $\rightarrow$ \texttt{0}
        \item \texttt{long} $\rightarrow$ \texttt{0l}
        \item \texttt{float} $\rightarrow$ \texttt{0.0f}
        \item \texttt{double} $\rightarrow$ \texttt{0.0d}
        \item \texttt{char} $\rightarrow$ \texttt{'\textbackslash u0000'}
        \item \texttt{boolean} $\rightarrow$ \texttt{false}
        \item Objekte $\rightarrow$ \texttt{null}
    \end{itemize}

\end{frame}

\begin{frame}[fragile]{Default-Konstruktor}
    \begin{exampleblock}{}
        \begin{lstlisting}[language=Java]
class Cat {}

// call default constructor
Cat garfield = new Cat();
        \end{lstlisting}
    \end{exampleblock}
\end{frame}


\begin{frame}[fragile]{Default-Konstruktor}
    \begin{exampleblock}{}
        \begin{lstlisting}[language=Java,basicstyle=\scriptsize]
class AmazonProduct {

    private double price;
    private int quantity;
    private String identifier;
    private boolean primeEligible;

    public void display() {
        System.out.println("Price: "            + this.price);
        System.out.println("Quantity: "         + this.quantity);
        System.out.println("Identifier: "       + this.identifier);
        System.out.println("Prime eligible: "   + this.primeEligible);
    }

}
        \end{lstlisting}

    \end{exampleblock}

\end{frame}

\begin{frame}[fragile]{Default-Konstruktor}
\begin{exampleblock}{}
    \begin{lstlisting}[language=Java]
(new AmazonProduct()).display();
    \end{lstlisting}
\end{exampleblock}

    \begin{exampleblock}{}
        \begin{lstlisting}
Price: 0.0
Quantity: 0
Identifier: null
Prime eligible: false
        \end{lstlisting}
    \end{exampleblock}
\end{frame}

\begin{frame}{Konstruktoren}
    \center
    \Huge{Ich will aber meinen eigenen Konstruktor definieren}
\end{frame}

\begin{frame}{Eigenen Konstruktor definieren}
    \begin{itemize}
        \item Konstruktoren sind doch \textbf{Methoden}!
        \pause
        \item \textit{Also Definition wie bei Methoden?}
        \pause
        \item Fast!
        \begin{itemize}
            \item Kein Rückgabetyp
            \item Nein, auch nicht \texttt{void}
            \item Name der Methode $==$ Name der Klasse
        \end{itemize}
    \end{itemize}
\end{frame}

\begin{frame}[fragile]{Eigenen Konstruktor definieren}
    \begin{exampleblock}{}
        \begin{lstlisting}[language=Java]
class Cat {

    private int lives;

    public Cat() {
        this.lives = 9;
    }

}
        \end{lstlisting}
    \end{exampleblock}
\end{frame}

\begin{frame}[fragile]{Eigenen Konstruktor definieren}
    \begin{exampleblock}{}
        \begin{lstlisting}[language=Java]
class Dog {

    private String name;

    public Dog(String name) {
        this.name = name;
    }

}
        \end{lstlisting}
    \end{exampleblock}
\end{frame}

\begin{frame}[fragile]{Eigenen Konstruktor aufrufen}
    \begin{itemize}
        \item \textit{Wie rufe ich nun meinen Konstruktor auf?}
    \end{itemize}

    \vspace{.2in}
    \begin{exampleblock}{}
        \begin{lstlisting}[language=Java,basicstyle=\huge]
Dog odie = new Dog("Odie");
        \end{lstlisting}
    \end{exampleblock}
\end{frame}

\begin{frame}[fragile]{Default-Konstruktor}
    \begin{itemize}
        \item \textit{Und was ist jetzt mit dem Default-Konstruktor von eben?}
        \item Sobald man einen eigenen Konstruktor definiert, \alert{\textbf{verschwindet der Default-Konstruktor}} von Java!
    \end{itemize}
    \begin{alertblock}{}
        \begin{lstlisting}[language=Java]
Dog odie = new Dog(); // geht nicht mehr
        \end{lstlisting}
    \end{alertblock}
    \vspace{.2in}
    \begin{itemize}
        \item \textit{Schade\dots}
    \end{itemize}
\end{frame}

\begin{frame}[fragile]{Eigene Konstruktoren}
    \begin{exampleblock}{}
        \begin{lstlisting}[language=Java]
public Dog() {}

public Dog(String name) {
    this.name = name;
}
        \end{lstlisting}
    \end{exampleblock}
    \begin{lstlisting}[language=Java]
Dog struppi = new Dog("Struppi");
Dog odie    = new Dog(); // geht jetzt wieder
    \end{lstlisting}
\end{frame}

\begin{frame}[fragile]{Eigene Konstruktoren}
    \begin{exampleblock}{}
        \begin{lstlisting}[language=Java]
public Dog() {
    this.name = "Odie";
}

public Dog(String name) {
    this.name = name;
}
        \end{lstlisting}
    \end{exampleblock}
    \begin{lstlisting}[language=Java]
Dog struppi = new Dog("Struppi");
Dog odie    = new Dog();
    \end{lstlisting}
\end{frame}

\begin{frame}[fragile]{Eigene Konstruktoren}
    \begin{exampleblock}{}
        \begin{lstlisting}[language=Java]
public Dog() {
    this("Odie");
}

public Dog(String name) {
    this.name = name;
}
        \end{lstlisting}
    \end{exampleblock}
    \begin{lstlisting}[language=Java]
Dog struppi = new Dog("Struppi");
Dog odie    = new Dog();
    \end{lstlisting}
\end{frame}

\begin{frame}[fragile]{Eigene Konstruktoren}
    \begin{exampleblock}{}
        \begin{lstlisting}[language=Java]
public Dog()            { ... }
public Dog(String name) { ... }
        \end{lstlisting}
    \end{exampleblock}
    \vspace{.2in}
    \begin{itemize}
        \item \textit{Hä\dots wieso geht das überhaupt?}
        \pause
        \item \textbf{Overloading!} (stay tuned)
    \end{itemize}
\end{frame}


\begin{frame}{Eigener Konstruktor: aber warum?}
    \begin{itemize}
        \item Ich will selbst bestimmen, mit welchen Werten ein neues Objekt initialisiert werden soll
        \item Welche Anfangswerte machen für meine Klasse Sinn?
        \item Anfangswerte könnten als Parameter übergeben werden
        \item Ich will prüfen, ob übergebene Werte auch gültig sind
    \end{itemize}
\end{frame}

\begin{frame}[fragile]{Eigener Konstruktor: aber warum?}
    \begin{exampleblock}{}
        \begin{lstlisting}[language=Java]
class Fraction {

    private int numerator;
    private int denominator;

    public int approximate() {
        return this.numerator / this.denominator;
    }

}
        \end{lstlisting}
    \end{exampleblock}
\end{frame}

\begin{frame}[fragile]{Eigener Konstruktor: aber warum?}
    \begin{exampleblock}{}
        \begin{lstlisting}[language=Java]
Fraction f = new Fraction();
int a = f.approximate();
        \end{lstlisting}
    \end{exampleblock}
\end{frame}

\begin{frame}{DIVIDE BY ZERO}
    \begin{figure}
        \includegraphics[scale=.35]{img/dividebyzero.jpg}
    \end{figure}
\end{frame}

\begin{frame}[fragile]{Eigener Konstruktor: aber aber aber}
    \begin{alertblock}{}
        Exception in thread "main" java.lang.ArithmeticException: / by zero
    \end{alertblock}
    \begin{exampleblock}{}
        \begin{lstlisting}[language=Java]
return this.numerator / this.denominator;
        \end{lstlisting}
    \end{exampleblock}
    \begin{itemize}
        \item \alert{\texttt{this.denominator} wurde vom Default-Konstruktor mit \texttt{0} initialisiert}
        \item Schlechte Idee als Startwert
        \item Aber: \textbf{Java kann das nicht ahnen!}
    \end{itemize}
\end{frame}

\begin{frame}{Eigener Konstruktor: aber aber aber}
    Der Entwickler der Klasse muss also dafür sorgen, dass\dots
    \begin{itemize}
        \item Attribute mit gültigen Werten initialisiert werden
        \item Objekte nicht im Laufe der Anwendung in einen ungültigen Zustand geraten
    \end{itemize}
\end{frame}

\begin{frame}[fragile]{Eigener Konstruktor: schon besser}
    \begin{exampleblock}{}
        \begin{lstlisting}[language=Java]
class Fraction {

    private int numerator;
    private int denominator;

    public Fraction() {
        this.numerator = 0;
        this.denominator = 1;
    }

}
        \end{lstlisting}
    \end{exampleblock}
\end{frame}


\section{Overloading}

\begin{frame}{Mehrere Konstruktoren}
    \begin{itemize}
        \item Java erlaubt es, \textbf{mehrere Konstruktoren} zu definieren
        \item Die Konstruktoren müssen sich in \textbf{Anzahl oder Typ der Parameter} unterscheiden
        \item Erzeugen eines Objekts auf verschiedene Weise möglich
    \end{itemize}
\end{frame}

\begin{frame}[fragile]{Mehrere Konstruktoren}
    \begin{lstlisting}[language=Java]
class Color {

    public Color(int red, int green, int blue) { ... }

    public Color(String hex) { ... }

    public Color() { ... }

}
    \end{lstlisting}
\end{frame}

\begin{frame}[fragile]{Mehrere Konstruktoren}
    \begin{lstlisting}[language=Java]
Color lime, blue, black;

lime    = new Color(180, 255, 32);
blue    = new Color("4ABAFF");
black   = new Color();
    \end{lstlisting}

\end{frame}


\begin{frame}{Overloading}
    \begin{itemize}
        \item Allgemein ist es erlaubt, \textbf{mehrere Methoden mit dem gleichen Namen} zu definieren
        \item Das nennt sich \textbf{Overloading}
        \item Methoden müssen sich in \textbf{Anzahl oder Typ der Parameter} unterscheiden
        \item Spezialfall: mehrere Konstruktoren
    \end{itemize}
\end{frame}


\begin{frame}[fragile]{Überladen von Methoden}
    \begin{exampleblock}{}
        \begin{lstlisting}[language=Java]
public void rotate(double angle) {
    // ...
}

public void rotate(Vector2D center, double angle) {
    // ...
}
        \end{lstlisting}
    \end{exampleblock}
\end{frame}

\appendix

\beginbackup

\begin{frame}{Getter und Setter}
    \begin{block}{}
        \textit{Getter} bezeichnet eine Methode zur \textbf{Abfrage} von Attributen
    \end{block}

    \begin{block}{}
        \textit{Setter} bezeichnet eine Methode zum \textbf{Setzen} von Attributen
    \end{block}
\end{frame}

\begin{frame}[fragile]{Getter und Setter}
    \begin{exampleblock}{}
        \begin{lstlisting}[language=Java]
public double getTemperature() {
    return this.temperature;
}
        \end{lstlisting}
    \end{exampleblock}
    \begin{exampleblock}{}
        \begin{lstlisting}[language=Java]
public void setTemperature(double temperature) {
    this.temperature = temperature;
}
        \end{lstlisting}
    \end{exampleblock}
\end{frame}

\begin{frame}{Statische Methoden und Attribute}
    \begin{itemize}
        \item Statische Methoden und Attribute sind \textbf{unabhängig vom Zustand eines Objekts}
        \item Schlüsselwort \texttt{static}
        \item In statischen Methoden\dots
        \begin{itemize}
            \item Kein Zugriff auf (nicht-statische) Attribute oder Methoden
            \item Kein \texttt{this}
        \end{itemize}
        \item Aufruf bzw. Zugriff über \textbf{Klassenname} anstatt Instanzvariable
    \end{itemize}
\end{frame}

\begin{frame}[fragile]{Statische Methoden und Attribute}
    \begin{exampleblock}{}
        \begin{lstlisting}[language=Java,basicstyle=\scriptsize]
class WeatherStation {

    public static double convertToFahrenheit(double celsius) {
        return (celsius * 9.0) / 5.0 + 32.0;
    }

}
        \end{lstlisting}
    \end{exampleblock}
    \begin{lstlisting}[language=Java]
double t = WeatherStation.convertToFahrenheit(8.0);
    \end{lstlisting}
\end{frame}

\begin{frame}[fragile]{Statische Methoden und Attribute}
    \begin{exampleblock}{}
        \begin{lstlisting}[language=Java,basicstyle=\scriptsize]
class WeatherStation {

    // count how many times we convert stuff
    public static int convertionCounter = 0;

    public static double convertToFahrenheit(double celsius) {
        WeatherStation.convertionCounter++;
        return (celsius * 9.0) / 5.0 + 32.0;
    }

}
        \end{lstlisting}
    \end{exampleblock}
\end{frame}

\begin{frame}{Methodenaufruf: Wert vs. Referenz}
    \begin{itemize}
        \item Objekte als Parameter werden \textbf{als Referenz} an die Methode übergeben
        \item Sonstige Parameter (primitive Datentypen) werden \textbf{als Wert} übergeben
        \vspace{.2in}
        \item \textbf{als Referenz}\dots
        \begin{itemize}
            \item Methode erhält \textbf{Referenz} auf Objekt im Speicher
            \item Objekt kann also von der Methode \textbf{verändert} werden
        \end{itemize}
        \item \textbf{als Wert}\dots
        \begin{itemize}
            \item Methode erhält \textbf{Kopie} der Variable
            \item Änderungen innerhalb der Methode haben \textbf{keine Auswirkung nach außen}
        \end{itemize}
    \end{itemize}
\end{frame}

\begin{frame}[fragile]{Methodenaufruf: Wert vs. Referenz}
    \begin{alertblock}{}
        \begin{lstlisting}[language=Java]
class BadMath {
    public static void square(int x) {
        x = x * x;
    }
}
        \end{lstlisting}
    \end{alertblock}

    \begin{lstlisting}[language=Java]
int number = 42;
BadMath.square(number);
System.out.println(number); // 42
    \end{lstlisting}
\end{frame}

\begin{frame}[fragile]{Methodenaufruf: Wert vs. Referenz}
    \begin{exampleblock}{}
        \begin{lstlisting}[language=Java]
class GoodMath {
    public static int square(int x) {
        return x * x;
    }
}
        \end{lstlisting}
    \end{exampleblock}

    \begin{lstlisting}[language=Java]
int number = 42;
number = GoodMath.square(number);
System.out.println(number); // 1764
    \end{lstlisting}
\end{frame}

\begin{frame}[fragile]{Methodenaufruf: Wert vs. Referenz}
    \begin{exampleblock}{}
        \begin{lstlisting}[language=Java]
class Dog {
    
    public void swapNames(Dog dog) {
        String tmp = this.name;
        this.name = dog.name;
        dog.name = tmp;
    }

    public void sayName() {
        System.out.println("Wauwau " + this.name);
    }

}
        \end{lstlisting}
    \end{exampleblock}
\end{frame}

\begin{frame}[fragile]{Methodenaufruf: Wert vs. Referenz}
    \begin{lstlisting}[language=Java]
Dog odie = new Dog("Odie");
Dog struppi = new Dog("Struppi");

struppi.swapNames(odie);

odie.sayName();
struppi.sayName();
    \end{lstlisting}

    \begin{exampleblock}{}
        \begin{lstlisting}
Wauwau Struppi
Wauwau Odie
        \end{lstlisting}
    \end{exampleblock}
\end{frame}


\begin{frame}{Fragen?}
    \begin{figure}
        \includegraphics[scale=.32]{img/Question-Rage-Face.jpg}
    \end{figure}
\end{frame}

\begin{frame}{Bis nächste Woche!}
    \begin{figure}
        \includegraphics[scale=.32]{img/dt161111.jpg}
        \caption{\footnotesize{dilbert.com}}
    \end{figure}
\end{frame}

\backupend

\end{document}
