%%
%% kit-prog-tutorial
%%
%% Slides for my Java programming tutorial at KIT using LaTeX beamer.
%%
%% Copyright (c) 2015-2016 YouniS Bensalah <younis.bensalah@gmail.com>
%%
%% This work is released to the public domain.
%% For the full copyright and license information, please view the LICENSE file.
%%

\documentclass[18pt]{beamer}

\usepackage{templates/beamerthemekit}

\usepackage[utf8]{inputenc}
\usepackage{hyperref}
\usepackage{listings}

\titleimage{road}

\newcommand{\tagline}{A closer look at Methods and Constructors}
\newcommand{\quotes}[1]{``#1''}

\title[Programmieren\hspace{2.5pt}--\hspace{2.5pt}\tagline]{\tagline}
\subtitle{Programmieren~\textbar~Tutorium 32}

\author{YouniS Bensalah}
\date{21. November 2016}

\institute{Chair for Software Design and Quality}

\usepackage[citestyle=authoryear,bibstyle=numeric,hyperref,backend=biber]{biblatex}
\addbibresource{templates/example.bib}
\bibhang1em

\begin{document}

% remove annoying figure prefix in caption
\setbeamertemplate{caption}{\raggedright\insertcaption\par}

\selectlanguage{english}

\begin{frame}
    \titlepage
\end{frame}

% \begin{frame}{Heute}
%     \tableofcontents
% \end{frame}

\section{Organisatorisches}

\begin{frame}{Prüfungsanmeldung}
    \begin{itemize}
        \item Im \textbf{Studiportal} anmelden
        \begin{itemize}
            \item \texttt{https://campus.studium.kit.edu}

        \end{itemize}
    \end{itemize}
    \begin{itemize}
        \item \textbf{CAS} $\rightarrow$ Anmeldung online (Info, InWi)
        \item \textbf{HISPOS} $\rightarrow$ Anmeldung entweder online oder über blauen Schein
        \begin{itemize}
            \item \alert{Sehr zeitnah im Studienbüro vorbeigehen!}
        \end{itemize}
    \end{itemize}
\end{frame}

\begin{frame}{Praktomat}
    \begin{itemize}
        \item Im \textbf{Praktomat} anmelden
        \begin{itemize}
            \item \url{https://praktomat.cs.kit.edu/2016_WS}
        \end{itemize}
        \item \alert{\textbf{Hier gebt ihr die Übungsaufgaben ab!!!!!!!1!!?????????????}}
        \item Login via KIT-Account (uxxxx)
        \item Zugang von außerhalb über VPN
        \begin{itemize}
            \item \url{https://www.scc.kit.edu/dienste/vpn.php}
        \end{itemize}
    \end{itemize}

\end{frame}

\begin{frame}{ILIAS}
    \begin{itemize}
        \item Im \textbf{ILIAS} anmelden
        \begin{itemize}
            \item \url{https://ilias.studium.kit.edu}
        \end{itemize}
        \item \textbf{Forum zu Vorlesung und Übungsaufgaben}
        \item Login via KIT-Account (uxxxx)
        \item \alert{Kurs \textit{Programmieren} beitreten!}
    \end{itemize}
\end{frame}



\section{Methoden}

\begin{frame}{Methoden}
    \begin{itemize}
        \item \textbf{Methoden beschreiben das Verhalten von Objekten}
        \pause
        \item Alle Objekte vom gleichen Typ können die gleichen Funktionen ausführen
        \begin{itemize}
            \item \textit{Alle Ampeln können von Rot auf Grün schalten}
        \end{itemize}
        \pause
        \vspace{.2in}
        \item Methoden können\dots
        \begin{itemize}
            \item Berechnungen durchführen
            \item Werte züruckgeben
            \item den Zustand eines Objekts verändern
        \end{itemize}
    \end{itemize}
\end{frame}

\begin{frame}{This}
    \begin{itemize}
        \item \texttt{this}
        \begin{itemize}
            \item Referenz auf aktuelles Objekt
            \item Sichtbar in jeder (nicht-statischen) Methode
            \item Vordefiniert!
            \item Erlaubt Zugriff auf Attribute und Methoden des aktuellen Objekts
            \item Explizite Unterscheidung zwischen Attributen und lokalen Variablen
        \end{itemize}
    \end{itemize}
\end{frame}

\begin{frame}[fragile]{This}
    \begin{exampleblock}{}
        \begin{lstlisting}[language=Java,basicstyle=\Huge]
this.name
        \end{lstlisting}
    \end{exampleblock}
\end{frame}


\begin{frame}[fragile]{This}
    \begin{exampleblock}{}
        \begin{lstlisting}[language=Java]
public void sayName() {
    System.out.println(this.name);
}
        \end{lstlisting}
    \end{exampleblock}
\end{frame}

\begin{frame}{Parameter}
    \begin{itemize}
        \item Ein Parameter
        \begin{itemize}
            \item ist eine Variable
            \item enthält an Methode übergebenen Wert
        \end{itemize}
        \item Eine Methode kann 0, 1 oder mehrere Parameter haben
    \end{itemize}
\end{frame}

\begin{frame}{Formaler vs. aktueller Parameter}
    \begin{itemize}
        \item \textbf{Formaler Parameter}
        \begin{itemize}
            \item Bezeichner, der in einer Methode verwendet wird
            \item $x$ ist formaler Parameter in $f(x) := e^x$
        \end{itemize}
        \vspace{.2in}
        \item \textbf{Aktueller Parameter}
        \begin{itemize}
            \item Tatsächlicher Wert, der an die Methode durch den Aufrufer übergeben wird
            \item $42$ ist aktueller Parameter in $f(42)$
        \end{itemize}
    \end{itemize}
\end{frame}

\begin{frame}[fragile]{Formaler vs. aktueller Parameter}
    \begin{exampleblock}{}
        \begin{lstlisting}[language=Java]
// something: formaler parameter
public void say(String something) {
    System.out.println(something);
}
        \end{lstlisting}
    \end{exampleblock}
    \begin{exampleblock}{}
        \begin{lstlisting}[language=Java]
String name = "Steve";
// name: aktueller parameter
student.say(name);
        \end{lstlisting}
    \end{exampleblock}
\end{frame}

\begin{frame}[fragile]{Rückgabewert}
    \begin{itemize}
        \item Eine Methode \textit{kann} einen Rückgabewert liefern
        \item Rückgabetyp \texttt{void} $\rightarrow$ \textbf{kein} Rückgabewert
    \end{itemize}
    \begin{exampleblock}{}
        \begin{lstlisting}[language=Java]
public int getHeight() {
    return this.height;
}

public void setWidth(int w) {
    this.width = w;
}
        \end{lstlisting}
    \end{exampleblock}
\end{frame}

\section{Konstruktoren}

\begin{frame}{Konstruktoren}
    \begin{itemize}
        \item Ein \textbf{Konstruktor} ist eine spezielle \textbf{Methode}, die beim \textbf{Erstellen eines neuen Objekts} (\texttt{new}) aufgerufen wird
        \item \textbf{Attribute} sollen \textbf{initialisiert} werden
        \item Das neue Objekt startet in \textbf{gültigem Zustand}
    \end{itemize}
\end{frame}

\begin{frame}{Default-Konstruktor}
    \begin{itemize}
        \item Java stellt für jede Klasse zunächst einen \textbf{Default-Konstruktor} zur Verfügung
        \item Keine Argumente!
        \item Alle Attribute werden mit \textit{Null} initialisiert
    \end{itemize}

\end{frame}

\begin{frame}[fragile]{Default Values}
    \textbf{Null?}
    \vspace{.2in}
    \begin{itemize}
        \item \texttt{byte}, \texttt{short}, \texttt{int} $\rightarrow$ \texttt{0}
        \item \texttt{long} $\rightarrow$ \texttt{0l}
        \item \texttt{float} $\rightarrow$ \texttt{0.0f}
        \item \texttt{double} $\rightarrow$ \texttt{0.0d}
        \item \texttt{char} $\rightarrow$ \texttt{'\textbackslash u0000'}
        \item \texttt{boolean} $\rightarrow$ \texttt{false}
        \item Objekte $\rightarrow$ \texttt{null}
    \end{itemize}

\end{frame}

\begin{frame}[fragile]{Default-Konstruktor}
    \begin{exampleblock}{}
        \begin{lstlisting}[language=Java]
class Cat {}

// call default constructor
Cat garfield = new Cat();
        \end{lstlisting}
    \end{exampleblock}
\end{frame}


\begin{frame}[fragile]{Default-Konstruktor}
    \begin{exampleblock}{}
        \begin{lstlisting}[language=Java,basicstyle=\scriptsize]
class AmazonProduct {

    private double price;
    private int quantity;
    private String identifier;
    private boolean primeEligible;

    public void display() {
        System.out.println("Price: "            + this.price);
        System.out.println("Quantity: "         + this.quantity);
        System.out.println("Identifier: "       + this.identifier);
        System.out.println("Prime eligible: "   + this.primeEligible);
    }

}
        \end{lstlisting}

    \end{exampleblock}

\end{frame}

\begin{frame}[fragile]{Default-Konstruktor}
\begin{exampleblock}{}
    \begin{lstlisting}[language=Java]
(new AmazonProduct()).display();
    \end{lstlisting}
\end{exampleblock}

    \begin{exampleblock}{}
        \begin{lstlisting}
Price: 0.0
Quantity: 0
Identifier: null
Prime eligible: false
        \end{lstlisting}
    \end{exampleblock}
\end{frame}

\begin{frame}{Konstruktoren}
    \center
    \Huge{Ich will aber meinen eigenen Konstruktor definieren!}
\end{frame}

\begin{frame}{Eigenen Konstruktor definieren}
    \begin{itemize}
        \item Konstruktoren sind doch \textbf{Methoden}!
        \pause
        \item \textit{Also Definition wie bei Methoden?}
        \pause
        \item Fast!
        \begin{itemize}
            \item Kein Rückgabetyp!
            \item Nein, auch nicht \texttt{void}
            \item Name der Methode $==$ Name der Klasse
        \end{itemize}
    \end{itemize}
\end{frame}

\begin{frame}[fragile]{title}
    \begin{exampleblock}{}
        \begin{lstlisting}[language=Java]
class Cat {
    private int lives;

    public Cat() {
        this.lives = 9;
    }
}
        \end{lstlisting}

    \end{exampleblock}

\end{frame}



\section{Overloading}

\begin{frame}{Overloading}
    body
\end{frame}

\appendix

\beginbackup

\begin{frame}{Fragen?}
    \begin{figure}
        \includegraphics[scale=.32]{img/Question-Rage-Face.jpg}
    \end{figure}
\end{frame}

\begin{frame}{Bis nächste Woche!}
    \begin{figure}
        \includegraphics[scale=.32]{img/dt161111.jpg}
        \caption{\footnotesize{dilbert.com}}
    \end{figure}
\end{frame}

\backupend

\end{document}
