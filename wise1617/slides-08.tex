%%
%% kit-prog-tutorial
%%
%% Slides for my Java programming tutorial at KIT using LaTeX beamer.
%%
%% Copyright (c) 2015-2016 YouniS Bensalah <younis.bensalah@gmail.com>
%%
%% This work is released to the public domain.
%% For the full copyright and license information, please view the LICENSE file.
%%

\documentclass[18pt]{beamer}

\usepackage{templates/beamerthemekit}

\usepackage[utf8]{inputenc}
\usepackage{hyperref}
\usepackage{listings}
\usepackage{xcolor}
\usepackage{colortbl}

\titleimage{road}

\definecolor{lime}{HTML}{8FFF53}

\newcommand{\tagline}{Interfaces and Generics}

\newcommand{\quotes}[1]{``#1''}

\title[Programmieren\hspace{2.5pt}--\hspace{2.5pt}\tagline]{\tagline}
\subtitle{Programmieren~\textbar~Tutorium 32}

\author{YouniS Bensalah}
\date{19. Dezember 2016}

\institute{Chair for Software Design and Quality}

\usepackage[citestyle=authoryear,bibstyle=numeric,hyperref,backend=biber]{biblatex}
\addbibresource{templates/example.bib}
\bibhang1em

\begin{document}

% remove annoying figure prefix in caption
\setbeamertemplate{caption}{\raggedright\insertcaption\par}

\selectlanguage{english}

\begin{frame}
    \titlepage
\end{frame}

% \begin{frame}{Heute}
%     \tableofcontents
% \end{frame}

\section{Interfaces}

\begin{frame}{Interfaces}
    \textbf{Interfaces} stellen Schnittstellen dar
    \begin{itemize}
        \item Ein Interface ist eine Sammlung von Methodensignaturen \textit{ohne} Implementierung
        \item Eine Klasse kann ein \textbf{Interface} implementieren
        \item Die implementierende Klasse muss dann \textbf{jede Methode} des Interfaces implementieren
        \item Eine Klasse kann auch \textbf{mehrere Interfaces} gleichzeitig implementieren
    \end{itemize}
\end{frame}

\begin{frame}[fragile]{Interfaces}
    \begin{exampleblock}{}
        \begin{lstlisting}[language=Java]
public interface Eatable {

    public void consume();

}
        \end{lstlisting}

    \end{exampleblock}

    \begin{exampleblock}{}
        \begin{lstlisting}[language=Java]
public class Orange implements Eatable {

    @Override
    public void consume() {
        System.out.println("Peel first, then eat.");
    }

}
        \end{lstlisting}

    \end{exampleblock}

\end{frame}

\begin{frame}[fragile]{Interfaces}

    Das geht auch\dots

    \begin{exampleblock}{}
        \begin{lstlisting}[language=Java]
public abstract class Fruit implements Eatable {

    @Override
    public abstract void consume();

}
        \end{lstlisting}

    \end{exampleblock}
\end{frame}


\begin{frame}[fragile]{instanceof-Operator}
    Mit \textbf{instanceof} kann geprüft werden, ob ein gegebenes Objekt von einem bestimmten Typ ist
    \begin{exampleblock}{}
        \begin{lstlisting}[language=Java,basicstyle=\scriptsize]
Fruit fruit;
// ...

if (fruit instanceof Orange) {
    System.out.println("This is an Orange.");
} else {
    System.out.println("This is a random Fruit.");
}
        \end{lstlisting}

    \end{exampleblock}

    \pause

    \alert{\textbf{instanceof} sollte sehr sparsam verwendet werden!}

\end{frame}

\begin{frame}[fragile]{Typecast}
    \begin{lstlisting}
(Type) variable
    \end{lstlisting}

    \vspace{.2in}
    \begin{itemize}
        \item Up-Cast
            \begin{itemize}
                \item ist immer erlaubt, da \textbf{is-a} Beziehung gilt
                \item wird bei Zuweisungen und Methodenaufrufen implizit ausgeführt
            \end{itemize}
        \item Down-Cast
        \begin{itemize}
            \item kann zu Laufzeitfehler führen
            \item ist immer explizit
            \item sollte vorher mit \textbf{instanceof} geprüft werden
        \end{itemize}
    \end{itemize}
\end{frame}

\section{Generics}

\begin{frame}{Generics}
    Die Idee bei Generics ist, dass man auch \textbf{Typen als Parameter} an Klassen und Methoden übergeben kann.
\end{frame}

\begin{frame}{Generics}
    \textbf{Typen als Parameter? Was bringt das?}
    \begin{itemize}
        \item Man kann zum Zeitpunkt der Implementierung noch unbekannte Typen durch Variablen ersetzen
        \item Container-Datentypen können somit unabhängig vom Typ der Daten sein
        \begin{itemize}
            \item Eine Liste muss nicht wissen, welche Art von Daten sie später mal enthält
            \item Das Prinzip von verketteten Listen funktioniert immer gleich, egal welchen Typ die einzelnen Elemente haben
        \end{itemize}
        \item Kein redundanter Code
    \end{itemize}

\end{frame}

\begin{frame}[fragile]{Generics: Syntax}

    \begin{exampleblock}{}
        \begin{lstlisting}[language=Java]
class Foo<T> { ... }

Foo<Bar> x = new Foo<Bar>();
        \end{lstlisting}

    \end{exampleblock}

\end{frame}

\begin{frame}[fragile]{Generics: Beispiel}
    \begin{lstlisting}[language=Java]
/**
 * Generic version of the Box class.
 * @param <T> the type of the value being boxed
 */
public class Box<T> {
    // T stands for "Type"
    private T t;

    public void set(T t) { this.t = t; }
    public T get() { return t; }
}
    \end{lstlisting}

\end{frame}

\begin{frame}[fragile]{Generics: Beispiel}
    \begin{lstlisting}[language=Java]
Box<Integer> integerBox = new Box<Integer>();
    \end{lstlisting}

\end{frame}

\begin{frame}{Generics: Naming Conventions}
    \begin{itemize}
        \item E - Element (used extensively by the Java Collections Framework)
        \item K - Key
        \item N - Number
        \item T - Type
        \item V - Value
    \end{itemize}
\end{frame}

\begin{frame}[fragile]{Generics: Bounded Type Parameters}
    \begin{itemize}
        \item Manchmal sinnvoll, den generischen Typ \textbf{einzuschränken}
        \item Beispiel: \textit{\texttt{T} soll bitte Unterklasse von \texttt{Animal} sein!}
    \end{itemize}

    \begin{lstlisting}[language=Java]
class Box<T extends Fruit> { ... }

Box<Integer> intBox = new Box<Integer>(); // nope
Box<Orange> orangeBox = new Box<Orange>(); // ok
    \end{lstlisting}

\end{frame}


\begin{frame}[fragile]{Generics: Wildcards}
    \begin{itemize}
        \item $?$ nennt sich Wildcard
        \item \texttt{$?$ extends Foo} kann jeder Typ sein, der \texttt{Foo} erweitert (oder \texttt{Foo} selbst)

    \end{itemize}
    \begin{lstlisting}[language=Java]
public static void foo(Box<? extends Fruit> b) { ... }
    \end{lstlisting}

    \begin{itemize}
        \item \texttt{b} könnte z.B. \texttt{Box<Orange>} oder \texttt{Box<Apple>} sein
        \item Aber auch \texttt{Box<Fruit>}
    \end{itemize}
\end{frame}

\begin{frame}[fragile]{Generics: Diamond}
    \begin{itemize}
        \item Compiler kann den Typ aus dem Kontext schließen
        \item Weniger Schreibarbeit
    \end{itemize}
    \begin{lstlisting}[language=Java]
Box<Integer> integerBox = new Box<>();
    \end{lstlisting}

\end{frame}

\appendix
\beginbackup

\begin{frame}{Checkstyle}
    \begin{itemize}
        \item Anzahl vorgegebener Regeln, die den Programmierer \quotes{zwingen}, sauberen und lesbaren Code zu schreiben
        \item Diese Regeln werden in einer Checkstyle-XML-Datei zusammengefasst und können automatisch überprüft werden
        \item Nicht nur der Compiler muss den Code verstehen !
        \item Gecheckt werden\dots
        \begin{itemize}
            \item Whitespaces
            \item Namenskonventionen
            \item JavaDoc
        \end{itemize}
        \item Siehe \textit{Checkstyle} in der \textit{Programmieren Wiki} im ILIAS\\
        \url{https://ilias.studium.kit.edu/goto.php?target=wiki_462045_Checkstyle}
    \end{itemize}

\end{frame}

\begin{frame}{Gleichheit von Objekten}
    \begin{itemize}
        \item In Java gibt es zwei Möglichkeiten, auf Gleichheit zu prüfen
        \begin{itemize}
            \item \texttt{"=="}\\
            Prüft auf Wert-Gleichheit

            \item \texttt{equals}\\
            Prüft auf inhaltliche Objekt-Gleichheit
        \end{itemize}
        \item Klasse kann \texttt{equals}-Methode \textbf{überladen} und Gleichheit von zwei Objekten selbst definieren
        \item \alert{Bei \textbf{Objekten} ist Wert-Gleichheit gerade \textbf{Gleichheit der Referenz}!}
    \end{itemize}
\end{frame}

\begin{frame}{Debugging}
    \textbf{Debugging}: Finden und Entfernen von Fehlern (Bugs) in einem Programm
\end{frame}

\begin{frame}{Debugging}
    Praktische Tipps:
    \begin{itemize}
        \item Code erneut durchlesen
        \item Verschiedene Eingaben testen und genau auf Resultat achten
        \item Inhalte an verschiedenen Stellen im Programm ausgeben lassen
        \item An geeigneten Stellen feste Werte reinschreiben und schauen was dann passiert
        \item Debugger verwenden (\alert{Live Demo})
        \item \dots
    \end{itemize}
\end{frame}

\begin{frame}{Tipp 12: Comment your code.}
    \begin{itemize}
        \item Kommentare beschreiben \textbf{Logik}, nicht Java-Syntax
        \item Javadoc verwenden!
        \begin{itemize}
            \item Kurz die Aufgabe der Klasse/Methode beschreiben
            \item Parameter (Typ und Semantik) (\texttt{@param})
            \item Rückgabewert (Typ und Semantik) (\texttt{@return})
            \item Exceptions? (\texttt{@throws})
        \end{itemize}
    \end{itemize}
\end{frame}

\begin{frame}[fragile]{Tipp 12: Comment your code.}
    \begin{itemize}
        \item Kommentare beschreiben Logik, nicht Java-Syntax.
    \end{itemize}
    \begin{columns}[c]
        \column{.5\textwidth}
        \begin{alertblock}{Not so good}
            \begin{lstlisting}[language=Java,basicstyle=\scriptsize]
// check if a bigger than b
if (a > b) {
    // check if a bigger than c
    if (a > c) {
        // returns a
        return a;
    }
    // returns c
    return c;
// check if b bigger than c
} else if (b > c) {
    // returns b
    return b;
}
// returns c
return c;
            \end{lstlisting}
        \end{alertblock}
        \column{.5\textwidth}
        \begin{exampleblock}{Much better}
            \begin{lstlisting}[language=Java,basicstyle=\scriptsize]
// return the maximum of a, b, and c
if (a > b) {
    if (a > c) {
        return a;
    }
    return c;
} else if (b > c) {
    return b;
}
return c;
            \end{lstlisting}
        \end{exampleblock}
    \end{columns}
\end{frame}

\begin{frame}[fragile]{Tipp 12: Comment your code.}
    \begin{itemize}
        \item Javadoc verwenden!
    \end{itemize}
    \begin{exampleblock}{Klasse}
        \begin{lstlisting}[language=Java,basicstyle=\tiny]
/**
 * Recursive descent parser for the recommend command.
 *
 * @version 1.1
 * @author YouniS Bensalah <younis.bensalah@gmail.com>
 */
class RecommendParser { ... }
        \end{lstlisting}
    \end{exampleblock}
\end{frame}

\begin{frame}[fragile]{Tipp 12: Comment your code.}
    \begin{itemize}
        \item Javadoc verwenden !
    \end{itemize}
    \begin{exampleblock}{Methode}
        \begin{lstlisting}[language=Java,basicstyle=\tiny]
/**
 * Parse the term and return its meaning, i.e., a set of Products.
 *
 * @param code The line containing the term.
 * @return The set of recommended Products.
 * @throws SyntaxException if the term did not match the specified syntax rules.
 * @throws InvalidNodeException if the required node does not exist.
 */
public Set<Product> parse(String code) throws SyntaxException, InvalidNodeException { ... }
        \end{lstlisting}
    \end{exampleblock}
\end{frame}

\begin{frame}{Tipp 12: Comment your code.}
    \begin{itemize}
        \item \textbf{How to Write Doc Comments for the Javadoc Tool}\\
        \url{http://www.oracle.com/technetwork/articles/java/index-137868.html}
    \end{itemize}
\end{frame}

\begin{frame}{Fragen?}
    \begin{figure}
        \includegraphics[scale=.6]{img/additionalquestions.jpg}
    \end{figure}
\end{frame}

\begin{frame}{Bis nächste Woche!}
    \begin{figure}
        \includegraphics[scale=2.5]{img/dt161218.jpg}
        \caption{\footnotesize{dilbert.com}}
    \end{figure}
\end{frame}

\backupend

\end{document}
